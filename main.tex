\documentclass{alon}
\usepackage[utf8]{inputenc}
\usepackage{graphicx}
\usepackage{amsmath}
\usepackage{caption}
\usepackage{subcaption}

\title{capitoloML}
\author{Tommaso Boccali }
\date{November 2020}

% Title:
%Machine Learning for Monte Carlo simulations

%Possible TOC:

%Introduction to Neural Networks (with emphasis on generative models)
% Neural networks and Deep Learning (2 pages)
% Convolutional networks (and recurrent neural networks ?) (1 page)
% Generative models (6 pages)
%  Auto-Encoders and Variational Auto-Encoders
%  Generative Adversarial Networks
%  Graph Networks
%Example of applications for Monte Carlo simulations (of interest in medical physics and beyond) (3 pages)
% Emulation of electromagnetic interaction models (?)
% Emulation of low energy nuclear interaction models
% Emulation of radiation-matter interactions
% Emulation of detector responses (?)
% Strong and weak points about Neural Networks applications for Monte Carlo simulations (3 pages)
% Speed, accuracy and reliability
% Response to unexpected (untrained) signals
% Unphysical responses and how to impose physical constraintshttps://www.overleaf.com/project/607e87da177bcefc2d52bb65


%
\begin{document}
\chapter{Machine Learning for Monte Carlo simulations}
Computing techniques loosely based on mimicking the behavior of the human brain are becoming more and more important in a vast range of applications.
Their utilization is not new, with studies based on simple Neural Netwoks~\cite{nn1,nn2,nn3} dating back at tleast tot he 80s; what is instead quite recent is the possibility to deploy efficient computing architectures, often specifically tailored to the tasks.
At the same time, the capability to deploy larger and larger system has triggered theoretical studies, driving to more solid bases and to  the definition of more complex and pecialized models.

In these chapter we will start with an introduction to the model most relevant for Monte Carlo simulations, followed by a selection of spplications. In the last part of the chapter, we will review the trong and weak poins about the utilization of Neural Networks applications for Monte Carlo simulations.

\section{Introduction to Neural Networks}
Neural Networks are a specific branch of the Artificial Intelligence (\emph{AI}) domain in computer science.
They get their inspiration from the fact that humans are able to fulfill complex tasks; hence, by replicating the low-level mechanisms of the human brain on computing systems, one can potentially construct high level algorithms with similar capabilities.

\subsection{The human brain}
Neglecting any functional description, the human brain can be described as an organ composed by neuron, glial cells, neural stem cells and blood vessels (Figure~\ref{fig:humanbrain}).
\begin{figure}[h]
    \centering
    \begin{subfigure}[b]{0.35\textwidth}
    \centering
    \includegraphics[width=0.4\textwidth]{images/humanbrain.png}
    \caption{A pictorial view of the human brain (from Wikipedia).}
    \label{fig:humanbrain}
    \end{subfigure}
    \hfill
        \begin{subfigure}[b]{0.55\textwidth}
        \includegraphics[width=0.8\textwidth]{images/Blausen_0657_MultipolarNeuron.png}
     \caption{The artificial neuron.}
     \label{fig:neuron}
        \end{subfigure}

\end{figure}
 With our current understanding, the neurons are the units performing basic "operations" within the human brain, and their aggregate response is generating the high-level behaviour typical of humans.
 A neuron, as sketched in Figure~\ref{fig:neuron}, is composed of three main units: a number of dendrites, the soma (the cell body), and an axion; the total size largely varies between different types of neurons; the neurons used for cognitive functions (as those in the grey matter of the brain) are usually short, XX $\mu$m.
Functionally, a neuron is able to generate an electric response on the axion (\emph{output}), depending on the electrical potential present at the synapses (\emph{inputs}) present on the dendrites, generating a quite low-level response mechanism. Neurons are \emph{chained} by connections between axions and dendrites, generating a mesh in which N neurons are connected via M synapses.
 The high-level response of the human brain to stimula is understood to come from the complexity of such mesh, with a standard human brain featuring $~10^{11}$ neurons each with $~7000$ synapses, for a total of $~10^{15}$ "connections".

 In literature various models of the neuron behavior have been proposed~\cite{neuronbe1, neuronbe2, neuronbe3}, %https://en.wikipedia.org/wiki/Biological_neuron_model
 here we will focus on the simplest yet most simple to implement in computer systems~\cite{artificialneuron} (see Figure~\ref{fig:artificialneuron}): %https://en.wikipedia.org/wiki/Artificial_neuron
 
 \begin{figure}[h]
    \centering
    \hfill
        \includegraphics[width=0.9\textwidth]{images/artificialneuron.png}
     \caption{The artificial neuron.}
     \label{fig:artificialneuron}

\end{figure}
 
 in this model, the \emph{output} $y$ signal at the axion is assumed to be a function of the \emph{inputs} $x_i$ via
 \begin{equation}
   y = f(\sum_{i=1}^{N} w_i x_i)
   \label{eq:artificialneuron}
 \end{equation}
where $w_i$ are weights defined by chemical potentials at the synapses, and the function $f$ wants to model the non linearity of response of biological neurons with the \emph{inputs}; on top of this, the function $f$ is needed in the mathematical model in order to allow the description of non linear phenomena~\cite{nonlinearitytheorem}. The percepton~\cite{perceptron}, one of the first models used in literature for Neural Networks, uses a very similar model, with a simplified $f$ function which is simply
\begin{equation}
  f(\vec{x})= \begin{cases}
                1 &  \text{if}\  \sum_{i=1}^{N} w_i x_i >0 \\
                0 &  \text{otherwise}
              \end{cases}
\end{equation}
Today, two small modifications are typical when using Neural Networks:
\begin{itemize}
\item the addition of a further synapse $x_0$ which is always 1, as a bias to the system; its weight is referred to as $x_0$ or $b$ () as in \emph{bias}.
\item the use of continuous $f$ non linear functions, as the logistic~\cite{logistic} or the hyperbolic~\cite{hyperbolic} functions.
\end{itemize}

Neural networks are obtained by combining multiple neurons in \emph{networks}, usually in a layered structure: one layer is used to map the inputs, a few/many layers are \emph{hidden}, and a single layer used to to map the outputs. On top of that, more complex neurons an be used, for example including a "memory" cell, or presenting a recurrent behavior by reusing its output as one of the inputs. A full description of all the type of neurons and networks is beyond the scope of this chapter; in the following, the ones most relevant to Monte Carlo simulations will be presented wit more detail. For reference, still, a complete classification of currently relevant neural networks is shown in Figure~\ref{fig:types}.
\begin{figure}[h]
    \centering
    \includegraphics[width=0.9\textwidth]{images/types.png}
    \caption{Types of neurons and neural networks currently relevant in literature (Copyright F. van Veen 2016).}
    \label{fig:types}
\end{figure}
As clearly visible in the figure, some network topologies have an high number of hidden layers. While the Universal Approximation Theorem~\cite{nonlinearitytheorem} states that under quite generic conditions, a single hidden layer between inputs and outputs should be enough in all cases, networks used in science during the last decade tend to be "deep" (i.e. with many hidden layers). This has multiple motivations: on one side, the theorem states that it is possible to have just one hidden layers, but does not state with how many neurons (and it tends to be a very large number); on the other side, a deep structure tend to be better human readable, with cascade sub networks with a more identifiable and logical role. Hence, relevant networks in today's science tend to be deep.

The typical utilization pattern for a majority of network topologies is to feed them as input a large set of data representing the problem of interest, be it a medical image, a set of features or any output from the instrumentation, and at the same time provide the "expected output" from a so-called training set. The network adjusts its internal free degrees of freedom (the weights $w_i^j$ in equation~\ref{eq:artificialneuron}, extended to the $j$ neurons in the various layers) to better reproduce the desired answers, via minimization procedures which can be either numerical or analytical.

What has just been described is the training process for \emph{supervised} Neural networks, which rely on an externally provided "truth" to adjust for optimal performance, without having any a-priori knowledge of the physical process they want to reproduce.
Other topologies describe instead the \emph{unsupervised} Neural Networks, in which the training process just implies the utilization of datasets without the need to provide the correct answers". Examples of such networks will be provided in Section~\ref{sec:unsupervised}.


\subsection{Convolutional networks}
Convolutional networks are a useful subset of neural networks, which exhibit peculiar characteristics of being space invariant with respect of the inputs.

They are particularly interesting in the realm of Monte Carlo simulations, since the space invariance is a valuable characteristics: for example, a shower from an hadronic particle into a material, when far from the edges, does not depend on the specific entry point

\subsection{Recurrent networks}

\subsection{Generative Models}
The networks popular up to 10 years ago were mostly useful during a decision process, such has categorizing inputs (signal vs background, for example) or counting and defining specific regions inside it (segmentation, counting of lesions, etc).

In order to be applied to Monte Carlo simulations, instead, the capability to produce ("generate") an output as close as possible to reality, or to a more standard algorithm. In order to do this, different network topologies and strategies for training are relevant.

%It needs to be highlighted that none of the methods described in this chapter have any intrinsic knowledge of the physics involved in the simulation processes. They are "trained" on external data, which needs to be accurate as possible. While to some extent it is possible to think of a training process using data, in practice the data samples and setups to ensure a generic training is huge, and not easy to obtain in practice. Hence, standard training samples are usually provided via "old school" detailed simulation toolkits, like Geant4~\cite{g4} or FLUKA~\cite{fluka}; it is important to notice that the advent of Machine Learning techniques for Monte Carlo simulations does not imply that the latter will be less needed, on the contrary it will remain fundamental to have  understood and well tuned reference tools to be used to train AI inspired tools.

\subsubsection{Auto-Encoders and Variational Auto-Encoders}
\label{sec:unsupervised}
The autoencoders are a family of unsupervised Neural Networks designed to learn a lower dimensional representation (say N dimensions) out of a set of data (with M dimensions, M \textgreater N). The N dimension representation ("latent space") can be seen as coming from a a 

The easiest form of autoencoder is one in which the number of inputs and outputs equals to M, and where there is a layer at dimensionality N. The training is obtained by forcing the network to reproduce as close a possible the input features at the output layer, thus requesting the N-dimension space to be as sufficient as possible to represent the M-dimension inputs (see Figure~\ref{fig:autoencoder}).

\begin{figure}[h]
     \centering
     \includegraphics[width=0.85\textwidth]{images/autoencoder.png}
     \caption{An autoencoder in its simplest form.}
     \label{fig:autoencoder}
 \end{figure}
 
 Autoencoders in such form are used for two distinct purposes:
 \begin{itemize}
     \item auto-discover in the inputs hidden symmetries or underlying correlations, which can be used, for example, in lossy compression scheme or to drive understanding on the inputs themselves;
     \item since the network is trained on a specific data sample, it will minimize the difference outputs vs inputs on that specific dateaset. Once the same network is presented with 
 \end{itemize}
 

\subsubsection{Generative Adversarial Networks}
\subsubsection{Graph Networks}





\section{Examples of applications for Monte Carlo simulations}
\label{sec:examples}
%guardare:

%~\cite{Sarrut2018} Sarrut D, Krah N, Badel JN, Létang JM. Learning SPECT detector angular response function with neural network for accelerating Monte-Carlo simulations. Phys Med Biol 2018;63. https://doi.org/10.1088/1361-6560/aae331.

%~\cite{Sarrut2019}
%Sarrut D, Krah N, Létang JM. Generative adversarial networks (GAN) for compact beam source modelling in Monte Carlo simulations. Phys Med Biol 2019;64:215004. https://doi.org/10.1088/1361-6560/ab3fc1.

%~\cite{Sadeghnejad-Barkousaraie2020}
%Sadeghnejad-Barkousaraie A, Bohara G, Jiang S, Nguyen D. A reinforcement learning application of guided Monte Carlo Tree Search algorithm for beam orientation selection in radiation therapy. ArXiv 2020. https://doi.org/10.1088/2632-2153/abe528.

%~\cite{Liu2020a} 
%Liu CC, Huang HM. A deep learning approach for converting prompt gamma images to proton dose distributions: A Monte Carlo simulation study. Phys Medica 2020;69:110–9. https://doi.org/10.1016/j.ejmp.2019.12.006.

%~\cite{Peng2019}
%Peng Z, Shan H, Liu T, Pei X, Zhou J, Wang G, et al. Deep learning for accelerating Monte Carlo radiation transport simulation in intensity-modulated radiation therapy. ArXiv 2019:1–8.

%\cite{Ma2020c}
%Ma S, Hu Z, Ye K, Zhang X, Wang Y, Peng H. Feasibility study of patient-specific dose verification in proton therapy utilizing positron emission tomography (PET) and generative adversarial network (GAN). Med Phys 2020;47:5194–208. https://doi.org/10.1002/mp.14443.

Guardare:
Applications of Artificial Intelligence in Particle Radiotherapy
https://arxiv.org/pdf/2102.03061.pdf




%\subsection{Emulation of electromagnetic interaction models}

\subsection{Emulation of radiation-matter interactions}
\label{subsec:interactions}
%tommaso
% computational cost of full fledged simulations; their not complete ability to match data
One of the most complex tasks in Monte Carlo simulations involving the use of detectors (medical apparatuses, particle physics detectors) lies in the dual need of being able to optimize the design before detector construction, and to simulate the behavior under working conditions after the setup has been prepared.
In both cases, unless the setup is very similar to existing detectors, extensive Monte Carlo simulations of the expected detector capabilities are the widely used solutions. Various such tools exist (Geant4\cite{g4}, Fluka\cite{fluka}, MCNPX\cite{MCNPX}), with different application regimes and specific utilization patterns. As a general rule, these implement iteratively basic radiation-matter low-level processes to a knowledge of the detector setup, including materials, geometry and  other experimental conditions; as such, they incur into two general limits:
\begin{itemize}
\item a scarce capability to be tuned to experimental results, by changing the basic modelling of the processes;
\item a large to very large need for computational resources, given the iteration oriented approach and the need to increase the level of iterations in order to obtain a better precision and adherence to data.
\end{itemize}

% Geant4, Fluka, MCNPX: https://doi.org/10.1063/1.2720459

Both limitations can in principle be surpassed via the use of Artificial Intelligence oriented tools.
In presence of experimental data, the response of the AI system can be tuned to that without any explicit modelling of the physics processes; speed can be vastly improved by the change from iteration-based computations to standard Deep Learning matrix algebra, with its intrinsic capabilities for high performance processing on, for example, GPU systems.

As an example we want to consider here CaloGan\cite{calogan}, an attempt to reproduce the details of radiation-matter interactions in the complex setup of segmented (3 layers) electromagnetic calorimeters.
A Generative Adversarial Network, as those presented in Section~\ref{subsec:gan}, is used in conjunction with an as-accurate-as-possible Geant4 simulation of the same experimental setup. The generator side accepts in input particles' 4-momenta, ad after the passage through quite standard convolutional (matching the detector response as 2-D images) and dense layers, the output is compared with detailed Geant4 simulations of particles with the same parameters.  The training optimizes the energy deposition per layer and per 2-D transverse cell, in a way in principle suited also for using real data in input. Results are very encouraging, even in an extreme detector scenario: not only the quantities of direct training are well reproduced, but also secondary and derived quantities like shower shapes are in most cases well described.

Results are shown in Figure~\ref{fig:calogan} for the explicit targets of the calculation (2-D layered images of the energy deposits, in the specific case of incoming positrons0.


\begin{figure}[h]
    \centering
    \begin{subfigure}[b]{0.55\textwidth}
    \centering
    \includegraphics[width=1\textwidth]{images/calogan1.png}
    \end{subfigure}
    \hfill
        \begin{subfigure}[b]{0.44\textwidth}
        \includegraphics[width=1\textwidth]{images/calogan2.png}
        \end{subfigure}
\caption{(left) Average e$^+$ GEANT4 shower (top), and average e$^+$
CALOGAN shower (bottom), with progressive calorimeter depth
(left to right). (right) Energy weighted shower depth (in a.u.) from CALOGAN and Geant4 detailed simulations. (From \cite{calogan}).}
        \label{fig:calogan}
\end{figure}


Interestingly, one can look into quantities derived from shower shapes, but not directly targeted by the GAN training step, like for example the energy weighted shower depth. As we will see in Section~\ref{subsec:speed}, in general one cannot assume these quantities will be perfectly reproduced; in this specific case, though, the agreement level is quite impressive.


A second similar attempt, applied to the not-yet existing CLIC proposed electromagnetic and hadronic calorimeter, is presented in~\cite{3dgan}, with the goal to directly reproduce 3-D signals in a highly granular calorimeter. The reference dataset, in absence of real data, has the form of Geant4 generated showers sampled in a 25x25x25 cells around the impinging particle.
% https://doi.org/10.1051/epjconf/201921402010
Figure~\ref{fig:3dgan}(left) shows the longitudinal shower shapes for 100 GeV electrons in the electromagnetic part of the calorimeter compared with detailed Geant4 simulations. The level of agreement is very satisfactory. Figure~\ref{fig:3dgan}(right) shows a pictorial rendering of the expected energy deposit by a 100 GeV electron in the calorimeter.



\begin{figure}[h]
    \centering
    \begin{subfigure}[b]{0.54\textwidth}
    \centering
\includegraphics [width=\textwidth]{images/3dgan.png}
    \end{subfigure}
    \hfill
        \begin{subfigure}[b]{0.44\textwidth}
        \includegraphics [width=\textwidth]{images/3dgan1.png}

        \end{subfigure}
        \caption{(left) Longitudinal shower shapes for 100 GeV electrons: GAN result is compared to full Geant4
simulation. The Z coordinate indicates the bin index in the longitudinal direction. (right) The three-dimensional representation of an energy shower created by a 100
GeV electron as generated by the GAN, using particle type as conditioning information. (From \cite{3dgan}).}
        \label{fig:3dgan}

\end{figure}



In sections~\ref{subsec:speed} and ~\ref{subsec:physical} we will discuss about the speed gain with respect to standard methods, and solutions and needs to prevent unphysical results.

\subsubsection{Emulation of detector responses}
Monte Carlo tools like Geant4 are designed to simulate, as accurately as possible, the energy deposition (in keV, for example) due to the passage of particle / radiation in the material of a detector. In real life, what a scientist measures is instead the response, as analog or digital signals, of a measuring device in which energy deposition is read and processed by some electronic boards. In classical systems, the simulation of radiation / matter must be followed by an ad-hoc simulation of the electronic readout system, in order to be compared with actual readings from a detector. In the case of AI inspired tools, this can be avoided by completely bypassing the "energy deposition" output results, and training the system directly with real or realistic (from the above ad-hoc simulation) signals from the electronic back-end. In ~\cite{mri}, for example, images from a real MRI apparatus are used for the training the system (see Figure~\ref{fig:elec}); the use of more classical approaches would be indeed more problematic since there is no practical way to measure (and validate) the output in terms of energy deposition in a running system.
%A similar approach is presented in~\cite{Sarrut20 18}, 

\begin{figure}[h]
    \centering

    \includegraphics[width=0.8\textwidth]{images/electronics.png}
    \caption{Difference between classical and AI inspired simulation of experimental setups.}
     \label{fig:elec}

\end{figure}

%ale: trovi tu una referenza decente questo
%tommaso

\subsection{Emulation of nuclear interaction models}
Nuclear reactions models are one of the most demanding part of a MC simulation in terms of running time. Despite the large running time, the models already available in toolkits made to develop MC simulation, such as Geant4, have shown severe limitations in reproducing experimental data below 100~MeV/u~\cite{g4-med}. Models developed by theoreticians for this energy domain can be interfaced with Geant4 with good results~\cite{blob-g4}, however their running time is even larger. Ciardiello et al.~\cite{blob-emulation} obtained encouraging preliminary results 
in emulating one of the state-of-the-art models for low energy nuclear reactions with a VAE. The model in question is BLOB (``Boltzmann-Langevin One Body'')~\cite{blob}, which simulates the first part of the nuclear reaction, from the contact of the two nuclei until the energy of the nucleons composing the fragments is balanced among them. The BLOB output is a PDF of finding a nucleon in a given position of the phase space. The authors trained a VAE in reproducing the BLOB prediction in the interaction of two $^{12}$C nuclei at 62~MeV/u. For this purpose, they discretized and reduced the dimensionality of the BLOB output to use 3D convolutional layers. In detail, the dimensionality reduction uses the fact that in the reaction in exam BLOB predicts at most three large fragments, i.e. larger than one nucleon. Therefore, they divided the PDF produced by BLOB in three PDFs, one per large fragment, and associated all the nucleon emitted in the first part of the reaction to one of these three large fragments. In this way they used the three colour channels of convolutional layers to represent each of the possible large fragments. In spite of controlling the generative part, they trained a classifier for the event impact parameter ($b$) jointly with the VAE itself. This joint training helps the VAE in learning a the task, given the large sparsity of the input, and force the latent space in being organise with respect to the impact parameter. Moreover, it will be possible to sample from the latent space deciding the impact parameter of each generation.
Figure \ref{fig:out} shows that the VAE is able to generate PDFs very similar to the input one when sampling a point nearby the position in the latent space where the input has been encoded.

%\begin{figure}[!bht]
%\centering
%\includegraphics [width=.7\textwidth]{images/latent3}    
%\caption{Representation of the VAE latent space described in the text. Each point is the encoding point of one training %distribution, the color scale represent the impact parameter of the event.}
%\label{fig:latent}
%\end{figure}

\begin{figure}[!bht]
\centering
\includegraphics [width=\textwidth]{images/generated}
%\includegraphics [width=.9\columnwidth]{images/cy1}
%\includegraphics [width=.9\columnwidth]{images/cz1}
\caption{Example of results obtained in generating with the VAE a distribution, in red, starting from a point sampled from the neighborhood of the point where the input distribution, in blue, is encoded. The three distributions are the projections on the three axis of 3D PDFs.}
\label{fig:out}
\end{figure}

%They conclude their work planning to enlarge the VAE latent space and train the VAE with different projectile energies and using different ions as projectile and target besides interfacing the VAE decoder with Geant4,
% so that it will be possible to use BLOB
% for simulating low energy nuclear interactions without its computational overhead

%\begin{figure}[!bht]
%\centering
%\includegraphics [width=.9\columnwidth]{images/test_reco}
%\caption{Double differential cross sections of alpha particle production in the reaction of $^{12}$C on a thin $^{12}$C target at 62~MeV/u as a function of the kinetic energy of the produced fragment for different angles. The experimental data, in black crosses, are from De Napoli et al.~\cite{DeNapoli:2012bs}. The continuous light blue lines show the BLOB predictions and the dashed magenta lines show the calculated values once encoded to reduce the PDF dimensionality and then decoded back.}
%\label{fig:testreco}
%\end{figure}

\subsection{A possible application of Graph Neural Networks}
To date, no specific study on Monte Carlo simulations of radiation / matter interactions using Graph Networks can be found in literature; still, it is clear from studies in other scientific domains suggest the potential of the technique. 
We want to cite here a study simulating the mechanical behavior of a fluid system, which shows capabilities of rendering the physics behind a complex system such as an ensemble of water small volumes\cite{acqua}; we expect similar GNN systems to become available for our research fields in short time.

\section{Strong and weak points about Neural Networks applications for Monte Carlo simulations}

\begin{thebibliography}{99}

\bibitem{nn1} 10.1016/S0370-2693(99)01288-5
\bibitem{nn2} DOI:10.1109/42.538937
\bibitem{nn3} Ben-Bassat, M.; Klove, K.L.; Weil, M.H. Sensitivity Analysis in Bayesian Classification Models: Multiplicative Deviations. IEEE Trans. Pattern Anal. Mach. Intell. 1980, PAMI-2, 261–266
\bibitem{neuronbe1}  DOI https://physoc.onlinelibrary.wiley.com/doi/abs/10.1113/jphysiol.1952.sp004764
\bibitem{neuronbe2}  ISBN:	051107817X 9780511078170 0511076606 9780511076602 0511075065 9780511075063 9780511815706 0511815700 1283329352 9781283329354 9780511203756 0511203756
\bibitem{nonlinearitytheorem} DOI https://doi.org/10.1016/0893-6080(91)90009-T
\bibitem{artificialneuron} \url{https://en.wikipedia.org/wiki/Artificial_neuron}
\bibitem{neuronlength} \url{https://aiimpacts.org/transmitting-fibers-in-the-brain-total-length-and-distribution-of-lengths/}
\bibitem{perceptron} doi:10.1037/h0042519
\bibitem{logistic} \url{https://en.wikipedia.org/wiki/Logistic_function}
\bibitem{lossfunctions} https://doi.org/10.1007/s40745-020-00253-5
\bibitem{cnnmedical}https://doi.org/10.1007/s10916-018-1088-1
\bibitem{compression}  DOI 10.1109/TBDATA.2021.3066151, IEEE
\bibitem{anomalymed} https://arxiv.org/pdf/2004.03271.pdf
\bibitem{anomalyhep} https://arxiv.org/abs/2005.01598
\bibitem{goodfellow} https://arxiv.org/abs/1406.2661

\bibitem{calogan} https://journals.aps.org/prd/pdf/10.1103/PhysRevD.97.014021

\bibitem{faces} \url{https://thispersondoesnotexist.com/}
\bibitem{ganmed} https://aapm.onlinelibrary.wiley.com/doi/10.1002/mp.13458
\bibitem{gnn} http://citeseerx.ist.psu.edu/viewdoc/download?doi=10.1.1.1015.7227&rep=rep1&type=pdf
\bibitem{clusterization} https://doi.org/10.1088/2632-2153/abbf9a


\bibitem{graphclustering} https://arxiv.org/abs/2008.06064
\bibitem{graphtracking} https://arxiv.org/pdf/2003.11603.pdf


\bibitem{g4} https://doi.org/10.1016/S0168-9002(03)01368-8
\bibitem{fluka} "The FLUKA Code: Developments and Challenges for High Energy and Medical Applications" T.T. Böhlen, F. Cerutti, M.P.W. Chin, A. Fassò, A. Ferrari, P.G. Ortega, A. Mairani, P.R. Sala, G. Smirnov and V. Vlachoudis, Nuclear Data Sheets 120, 211-214 (2014)
\bibitem{MCNPX} Status of the MCNPX Transport Code Advanced Monte Carlo for Radiation Physics, Particle Transport Simulation and Applications, 2001 ISBN : 978-3-642-62113-0 H. G. Hughes, M. B. Chadwick, R. K. Corzine, Show All (15)

\bibitem{vae} Auto-Encoding Variational Bayes, arXiv:1312.6114, D.P. Kingma and M. Welling, (2014)
\bibitem{bvae} An introduction to variational autoencoders, Foundations and Trends in Machine Learning: Vol. 12: No. 4, pp 307-392. http://dx.doi.org/10.1561/2200000056 (2019). arXiv:1906.02691.

\bibitem{blob} Bifurcations in Boltzmann-Langevin one body dynamics for fermionic systems, P.Napolitani and M.Colonna, 726 (2013) 382–386.

\bibitem{blob-g4} Preliminary results coupling “Stochastic Mean Field” and “Boltzmann-Langevin one body” models with Geant4. C. Mancini-Terracciano et al. Physica Medica, 67:116–122, (2019).

\bibitem{blob-emulation} Preliminary results in using deep
learning to emulate BLOB, a nuclear interaction model. A Ciardiello et al. Physica Medica, 73:65–72 (2020).

\bibitem{g4-med} Report on G4-Med, a Geant4 benchmarking system for medical physics applications developed by the Geant4 Medical Simulation Benchmarking Group. P. Arce et al. Medical Physics (2020). doi: 10.1002/mp.14226.


\bibitem{Sarrut2019}
Sarrut D, Krah N, Létang JM. Generative adversarial networks (GAN) for compact beam source modelling in Monte Carlo simulations. Phys Med Biol 2019;64:215004. https://doi.org/10.1088/1361-6560/ab3fc1.

\bibitem{Sarrut2018}
Sarrut D, Krah N, Badel JN, Létang JM. Learning SPECT detector angular response function with neural network for accelerating Monte-Carlo simulations. Phys Med Biol 2018;63. https://doi.org/10.1088/1361-6560/aae331.


\bibitem{Sadeghnejad-Barkousaraie2020}
Sadeghnejad-Barkousaraie A, Bohara G, Jiang S, Nguyen D. A reinforcement learning application of guided Monte Carlo Tree Search algorithm for beam orientation selection in radiation therapy. ArXiv 2020. https://doi.org/10.1088/2632-2153/abe528.

~\bibitem{Sadeghnejad-Barkousaraie2020}
Sadeghnejad-Barkousaraie A, Bohara G, Jiang S, Nguyen D. A reinforcement learning application of guided Monte Carlo Tree Search algorithm for beam orientation selection in radiation therapy. ArXiv 2020. https://doi.org/10.1088/2632-2153/abe528.

~\bibitem{Liu2020a} 
Liu CC, Huang HM. A deep learning approach for converting prompt gamma images to proton dose distributions: A Monte Carlo simulation study. Phys Medica 2020;69:110–9. https://doi.org/10.1016/j.ejmp.2019.12.006.

~\bibitem{Peng2019}
Peng Z, Shan H, Liu T, Pei X, Zhou J, Wang G, et al. Deep learning for accelerating Monte Carlo radiation transport simulation in intensity-modulated radiation therapy. ArXiv 2019:1–8.


\end{thebibliography}

\end{document}
