\section{Example of applications for Monte Carlo simulations}
\subsection{Emulation of electromagnetic interaction models}
\subsection{Emulation of low energy nuclear interaction models}
\subsection{Emulation of radiation-matter interactions}
%tommaso
% computational cost of full fledged simulations; their not complete ability to match data
One of the most complex tasks in Monte Carlo simulations involving the use of detectors (medical apparatuses, particle physics detectors) lies in the dual need of being able to optimize the design before detector construction, and to simulate the behavior under working conditions after the setup has been prepared.
In both cases, unless the setup is very similar to existing detectors, extensive Monte Carlo simulations of the expected detector capabilities are the widely used solutions. Various such tools exist (Geant4\cite{g4}, Fluka\cite{fluka}, MCNPX\cite{MCNPX})
% Geant4, Fluka, MCNPX: https://doi.org/10.1063/1.2720459
, with different application regimes and specific utilization patterns. as a general rule, these implement iteratively basic radiation-matter low-level processes to a knowledge of the detector setup, including materials, geometry and  XXX; as such, they incur into two general limits:
\begin{itemize}
\item a scarce capability to be tuned to experimental results, if not by changing the basic modelling of the processes;
\item a varge to very large need for computational resources, given the iteration oriented approach and the need to increase the level of iterations in order to obtain a better precision and adherence to data.
\end{itemize}

Both limitations can in principle be overridden via the use of Artificial Intelligence oriented tools.
In presence of experimental data, the response of the AI system can be tuned to that without any 


\subsection{Emulation of detector responses}
%tommaso
