\chapter{Machine Learning for Monte Carlo simulations}
\chapterinitial{C}{omputing} techniques loosely based on mimicking the behavior of the human brain are becoming more and more important in a vast range of applications.
Their utilization is not new, with studies based on simple Neural Networks (see for example~\cite{nn1,nn2,nn3}) dating back at least to the 80s; what is instead quite recent is the possibility to deploy efficient computing architectures, often specifically tailored to the tasks
% nn1: 10.1016/S0370-2693(99)01288-5
% nn2: DOI:10.1109/42.538937
%nn3: Ben-Bassat, M.; Klove, K.L.; Weil, M.H. Sensitivity Analysis in Bayesian Classification Models: Multiplicative Deviations. IEEE
Trans. Pattern Anal. Mach. Intell. 1980, PAMI-2, 261–266
At the same time, the capability to deploy larger and larger system has triggered theoretical studies, driving to more solid bases and to  the definition of more complex and specialized models.

In these chapter we will start with an introduction to the model most relevant for Monte Carlo simulations, followed by a selection of applications. In the last part of the chapter, we will review the strong and weak points about the utilization of Neural Networks applications for Monte Carlo simulations.
